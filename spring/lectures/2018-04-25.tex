\begin{flalign*}
	& (1) \begin{cases}
			\dot q = \pd{H}{P} \\
			\dot p = - \pd{H}{q} \\
	\end{cases} \qquad (2) \begin{cases}
		\tilde q = \tilde q(q,\; p,\; t) & (2.1) \\
		\tilde p = \tilde p(q,\; p,\; t) & (2.2) \\
	\end{cases} \qquad \det \pd{(\tilde q, \tilde p)}{(q, p)} \neq 0 &\\
	& (3)\;\;c((p, dq) - Hdt) = (\tilde p, d\tilde q) - \tilde Hdt + dF(q,\; p,\; t) &\\
\end{flalign*}

\subsection{Свободные канонические преобразования}
\begin{df}
	Преобразование вида $(2)$ --- своободное, если $\det \pd{\tilde q}{p} \neq 0$.
\end{df}
\begin{flalign*}
	& (2.1),\; \det \pd{\tilde q}{p} \neq 0 \Rightarrow p = p(q,\; \tilde q,\; t) &\\
	& (2.2),\; p = p(q,\; \tilde q,\; t) \Rightarrow \tilde p = \tilde p(q,\; \tilde q, t) &\\
	& F(q,\; p(q,\; \tilde q,\; t),\; t) = S(q,\; \tilde q,\; t) &\\
	& (3) \Rightarrow c((p, dq) - Hdt) = (\tilde p, d\tilde q) - \tilde Hdt + \left( \pd{S}{q}, dq \right) + \left( \pd{S}{\tilde q}, d\tilde q \right) + \pd{S}{t}dt &\\
	& \begin{cases}
		\pd{S}{q} = cp \\
		\pd{S}{\tilde q} = -\tilde p \\
		\tilde H = cH + \pd{S}{t}
	\end{cases} &\\
	& \det \pd{\tilde q}{p} \neq 0 \Leftrightarrow \det \pd{p}{\tilde q} \neq 0 \Leftrightarrow \det \frac{\partial^2 S}{\partial \tilde q \partial q} \neq 0 \quad (4) &\\
\end{flalign*}
\begin{ass}
	Свободное преобразование канонично тогда, и только тогда, когда $\exists \Phi,\; S(q,\; \tilde q, t),\; c\neq 0$:
	\[
		\begin{cases}
			\pd{S}{q} = cp \\
			\pd{S}{\tilde q} = -\tilde p \\
		\end{cases}.
	\]
	Тогда, если преобразование канонично, гамильтонова система в новых переменных имеет вид:
	\[
		\tilde H = \left.\left( cH(q,\;p,\;t) + \pd{S(q,\; \tilde q,\; t)}{t} \right)\right|_{\begin{array}{l}
			q = q(\tilde q,\; \tilde p,\; t) \\
			p = p(\tilde q,\; \tilde p,\; t) \\
		\end{array}}
	\]	
\end{ass}
\begin{ntc}
	Функция $S(q,\; \tilde q,\; t)$, удовлетворяющая условию $(4)$, однозначно определена преобразованием заданной валентности.
\end{ntc}
\begin{proof}
	\begin{flalign*}
		& \pd{S(q,\; \tilde q,\; t)}{q} = cp,\; \det \frac{\partial^2 S}{\partial \tilde q \partial q} \neq 0 \Rightarrow \tilde q = \tilde q(q,\; p,\; t) &\\
		& \tilde p = - \pd{S(q,\; \tilde q,\; t)}{\tilde q},\; \tilde q = \tilde q(q,\; p,\; t) \Rightarrow \tilde p = \tilde p(q,\; p,\; t) &\\
	\end{flalign*}
\end{proof}
\begin{xmp}
	\begin{flalign*}
		& \tilde q = q,\; \tilde p = p &\\
		& \det \pd{q^2}{p} = 0 \text{ --- не свободное.}
	\end{flalign*}
\end{xmp}
\begin{ntc}
	Вместо $(q,\; \tilde q,\; t)$ можно рассматривать $(q,\; \tilde p,\; t)$, $(\tilde q,\; p,\; t)$, $(p,\; \tilde p,\; t)$.
\end{ntc}
Рассмотрим, например, $(q,\; \tilde p,\; t)$:
\begin{flalign*}
	& (2) \text{ --- полусвободное, если } \det \pd{\tilde p}{p} \neq 0 &\\
	& \tilde p = \tilde p(q,\; p,\; t),\; \det \pd{\tilde p}{p} \neq 0 \Rightarrow p = p(q,\; \tilde p,\; t),\; \tilde q = \tilde q(q,\; \tilde p,\; t) &\\
	& c(p,dq) - cHdt = (\tilde p, d\tilde q) - \tilde Hdt + dF &\\
	& (\tilde p, d\tilde q) = d(\tilde p, \tilde q) - (\tilde q, d\tilde p),\; F(q,\; p,\; t) + (\tilde p, \tilde q) = S_1(q,\; \tilde p,\; y) \Rightarrow &\\
	& \Rightarrow c(p,dq) - cHdt = -(\tilde q, d\tilde p) - \tilde Hdt + \left( \pd{S}{q},dq \right) + \left( \pd{S}{\tilde p}, d\tilde p \right) + \pd{S}{t}dt &\\
	& \begin{cases}
		\pd{S_1}{q} = cp \\
		\pd{S_1}{\tilde p} = \tilde q &\\
		\tilde H = cH + \pd{S_1}{t}
	\end{cases}
\end{flalign*}
\begin{xmp}
	\begin{flalign*}
		& p = \tilde p,\; q = \tilde q,\; c = 1 &\\
		& \begin{cases}
			\pd{S_1(q,\; \tilde p,\; t)}{q} = \tilde p \\
			\pd{S_1}{\tilde p} = q \\
		\end{cases} \qquad S_1 = (q,\; \tilde p) &\\
	\end{flalign*}
\end{xmp}

\subsection{Уравнение Гамильтона-Якоби. Метод Якоби}
\begin{flalign*}
	& \tilde H = 0 \qquad \begin{cases}
		\dot{\tilde q} = \pd{\tilde H}{\tilde p} \\
		\dot{\tilde p} = -\pd{\tilde H}{\tilde q} \\
	\end{cases} \qquad \begin{array}{l}
		\tilde q = \alpha = const \\
		\tilde p = \beta = const \\
	\end{array} &\\
\end{flalign*}
Будем искать унивалентное свободное преобразование, переводящее гамильтониан в ноль.
\begin{flalign*}
	& \pd{S}{t} + H\left( q,\; \pd{S}{q},\; t \right) = 0 \quad (*) &\\
\end{flalign*}
Соответствует системе с гамильтонианом $H(q,\; p,\; t)$.
\begin{xmp}[Математический маятник]
	\begin{flalign*}
		& L = \frac{ml^2}{2}\dot \varphi^2 + mgl\cos\varphi &\\
		& \text{Т.к. } L = \frac{1}{2}(A(q)\dot q,\; \dot q) - \Pi(q)\text{, то } H = \frac{1}{2}(A^{-1}(q)\dot p,\; \dot p) + \Pi(q) &\\
		& H = \frac{p\varphi^2}{2ml^2} - mgl\cos\varphi &\\
		& \pd{S}{t} + \frac{1}{2ml^2}\left( \pd{S}{\varphi} \right)^2 - mgl\cos\varphi = 0&\\
	\end{flalign*}
\end{xmp}
\begin{df}
	Полным интегралом уравнения Гамильтона-Якоби называется функция $S(q,\; \alpha,\; t)$ такая, что
	\begin{enumerate}
		\item удовлетворяет уравнению Гамильтона-Якоби,
		\item $\det \frac{\partial^2 S}{\partial \alpha \partial q} \neq 0$.
	\end{enumerate}
\end{df}
\begin{teo}[Гамильтона-Якоби]
	Если известен полный интеграл Гамильтона-Якоби, то общее решения соответствующих уравнений Гамильтона определяется из соотношения
	\[
		\pd{S}{q} = p,\; \pd{S}{\alpha} = -\beta \Rightarrow \begin{cases}
			q = q(\alpha,\; \beta,\; t) \\
			p = p(\alpha,\; \beta,\; t) \\
		\end{cases}.
	\]
\end{teo}
\begin{ntc}
	Можно заменить $\beta$ на $-\beta$.
\end{ntc}
\begin{xmp}[Движение точки по прямой]
	\begin{flalign*}
		& x(0) = x_0,\; \dot x(0) = \dot x_0 &\\
		& L = \frac{m}{2}\dot x^2 \quad H = \frac{p_x^2}{2m} &\\
		& \pd{S}{t} + \frac{1}{2m}\left( \pd{S}{x} \right)^2 = 0,\; S = S(x,\; \alpha,\; t) &\\
		& S = \frac{m(x + \alpha)^2}{2t},\; \pd{S}{t} = - \frac{m(x+\alpha)^2}{2t^2} &\\
		& \pd{S}{x} = \frac{m(x + \alpha)}{t},\; \frac{\partial^2 S}{\partial \alpha \partial x} = \frac{m}{t} \neq 0 &\\
		& S \text{ --- полный интеграл уравнения } (*) &\\
		& \begin{cases}
			p_x = \pd{S}{x} = \frac{m(x + \alpha)}{t} \\
			- \beta = \pd{S}{\alpha} = \frac{m(x + \alpha)}{t} \\
		\end{cases} \qquad
		\begin{cases}
			p_x = - \beta \\
			x = -\frac{\beta}{m}t - \alpha \\
		\end{cases} \qquad
		\begin{cases}
			p_x = mx_0^2 \\
			x = \dot x_0 t + x_0
		\end{cases} &\\
		& \begin{cases}
			p_x(0) = m\dot x_0 = -\beta \\
			x(0) = x_0 = -\alpha \\
		\end{cases} \qquad
		\begin{cases}
			\beta = -m \dot x_0 \\
			\alpha = -x_0 \\
		\end{cases} &\\
		& S = S_1(x,\; \alpha) + S_2(t,\; \alpha) &\\
		& S = -\alpha t + S_1(x,\; \alpha) \rightarrow (*) &\\
		& -\alpha + \frac{1}{2m}\left( \pd{S_1}{x} \right)^2 = 0 \qquad \pd{S_1}{x} = \pm \sqrt{2m\alpha} \qquad S_1 = \pm x\sqrt{2m\alpha} &\\
		& \text{Тогда } S = -\alpha t \pm x\sqrt{2m\alpha} &\\
		& \text{<<$+$>> в области } p_x \geqslant 0\text{, <<$-$>> в области } p_x < 0 &\\
	\end{flalign*}
\end{xmp}
Рассмотрим натуральную систему с одной степенью свободы.
\begin{xmp}
	\begin{flalign*}
		& L = \frac{1}{2}a(q)\dot q^2 - \Pi(q) &\\
		& H = \frac{p^2}{2a(q)} + \Pi(q) &\\
		& \pd{S}{t} + \frac{1}{2a(q)}\left( \pd{S}{q} \right)^2 + \Pi(q) = 0 &\\
		& S = -h(\alpha)t + S_1(q,\; \alpha): &\\
		& -h(\alpha) + \frac{1}{2a(q)}\left( \pd{S_1}{q} \right)^2 + \Pi(q) = 0 &\\
		& \pd{S_1}{q} = \pm\sqrt{2a(q)(h(\alpha) - \Pi(q))} &\\
		& S = -h(\alpha)t \pm \int\limits_{q_0(\alpha)}^{q}\sqrt{2a(\xi)(h(\alpha) - \Pi(\xi))}d\xi &\\
		& \det \frac{\partial^2 S}{\partial q \partial \alpha} \neq 0 &\\
		& p = \pd{S}{q} = \pm \sqrt{2a(q)(h(\alpha) - \Pi(q))} &\\
		& -\beta = \pd{S}{\alpha} = -\frac{dh}{d\alpha}t \pm \int\limits_{q_0(\alpha)}^{q_0}\frac{d\xi \sqrt{a(\xi)}}{\sqrt{(h(\alpha) - \Pi(\xi))}} \mp \underbrace{\frac{dq_0}{d\alpha}}_0\underbrace{\sqrt{2a(q)(h(\alpha) - \Pi(q))}}_0 &\\
	\end{flalign*}
\end{xmp}

\subsection{Метод разделения переменных}
\begin{teo}[об отделении времени]
	Если $\pd{H}{t} = 0$, то $S(q,\; \alpha,\; t)$ --- полный интеграл уравнения
	\[
		\pd{S}{t} + H\left(q,\; \pd{S}{q}\right) = 0 \Leftrightarrow S = -h(\alpha)t + S_1(q,\; \alpha),
	\]
	где $S_1$ --- полный интеграл уравнения $H\left( q,\;\pd{S_1}{q} \right) = h(\alpha) \left( \text{т.е.} \det \frac{\partial^2 S_1}{\partial q, \partial \alpha} \neq 0 \right)$.
\end{teo}
\begin{proof}
	\begin{flalign*}
		& S = S_0(t,\; \alpha) + S_1(q,\; \alpha) &\\
		& \pd{S}{t} = \pd{S_0}{t},\; \pd{S}{q} = \pd{S_1}{q} &\\
		& -\frac{dS_0(t, \alpha)}{dt} \equiv H\left( q,\; \pd{S_1(q,\; \alpha)}{2q} \right) \equiv h(\alpha) &\\
		& \begin{cases}
			\pd{S_0}{t} = -h(\alpha) \Leftrightarrow S_0 = -h(\alpha)t + C(\alpha),\; C(\alpha) = const \\
			H\left( q,\; \pd{S_1}{q} \right) = h(\alpha) \\
		\end{cases} &\\
		& \frac{dS_0}{d\alpha} = -\frac{dh}{d\alpha}t + \frac{dC}{d\alpha} \neq 0,\; \frac{dh}{d\alpha} \neq 0 &\\
		& \det \frac{\partial^2 S_1}{\partial q \partial \alpha} \neq 0 \Leftrightarrow \det \frac{\partial^2 S}{\partial q \partial \alpha} \neq 0 &\\
	\end{flalign*}
\end{proof}

\begin{flalign*}
	& H = H(q,\; p,\; t) &\\
	& \pd{S}{t} + H(q,\; \pd{S}{q},\; t) = 0 &\\
	& \det \frac{\partial^2 S}{\partial q \partial \alpha} \neq 0,\; \alpha = const. &\\
\end{flalign*}

\begin{teo}[Об отделении координаты]
	Если $H = H(f(q_1,\; p_1),\; q_2,\ldots,q_n,\; p_2,\ldots,p_n,\; t)$, где $\pd{H}{f} \neq 0$, $\pd{f}{p_q} \neq 0$, то функция 
	\[
		S(q,\;\alpha) = S_1(q,\; \alpha_1) + S_2(q_2,\ldots,q_n,\; p_1,\ldots,p_n, t)
	\]
	является полным интегралом соответствующего уравнения Гамильтона-Якоби $\Leftrightarrow$
	\begin{enumerate}
		\item $S_1(q_1,\; \alpha_1)$ --- решение $f\left(q_1, \pd{S_1}{q_1}\right) = \alpha_1$;
		\item $S_2(q_1,\; \ldots, t)$ --- полный интеграл уравнения \[
			\pd{S_2}{t} + H\left( \alpha_1,\; q_2,\ldots,q_n,\; \pd{S_2}{q_2},\ldots,\pd{S_2}{q_n},\; t \right) = 0.
		\]	
	\end{enumerate}
\end{teo}
\begin{proof}
	\ovalbox{$\Rightarrow$}
	\begin{flalign*}
		& \pd{S}{t} + H\left( f\left(q_1,\; \pd{S}{q_1}\right),\; q_2,\ldots,q_n,\; \pd{S}{q_2},\ldots,\pd{S}{q_n},\; t \right) &\\
		& \det \frac{\partial^2 S}{\partial q \partial \alpha} \neq 0 &\\
		& S = S_1 + S_2 \qquad \pd{S_2}{t} + H\left( f\left(q_1, \pd{S_1}{q_1}\right), q_2,\ldots q_n,\; \pd{S_2}{q_2},\ldots \pd{S_2}{q_n}, t \right) = 0 &\\
		& \det \frac{\partial^2 S}{\partial q \partial\alpha} = \frac{\partial^2 S_1}{\partial q_1\partial\alpha}\cdot \det\left( \frac{\partial S_2}{\partial q_i \partial q_j} \right) \neq 0 &\\
		& (1)\; \pd{H}{f} = 0 \Rightarrow f\left( q_1, \frac{dS_1}{dq_1} \right) = F(q_2,\ldots,q_n,\; \pd{S_2}{q_2},\ldots,\pd{S_2}{q_n},\; t) &\\
		& S_1 = S_1(q_1,\; \alpha_1)\pd{S_2}{q_1} = 0 \Leftrightarrow f\left( q_1,\; \frac{dS_1}{dq_1} \right) = g(\alpha_1) = \tilde \alpha_1 &\\
		& \left.\begin{array}{l}
			\pd{f}{p_1} \neq 0 \Rightarrow \frac{dS_1}{dq_1} = G(q_1,\; \tilde \alpha_1) \\
			\frac{\partial^2 S_1}{\partial q_1\partial \alpha_1} \neq 0 \\
		\end{array}\right| \Rightarrow \alpha_1 = \tilde G(q_1,\; \tilde\alpha_1),\; 0 = \pd{\tilde G}{q_1} \Rightarrow \alpha_1 = \tilde g(\tilde \alpha_1) \Rightarrow 1. &\\
		& \left.\begin{array}{l}
		1.,\; (1) \Rightarrow \pd{S_2}{t} + H\left( \alpha_1,\; q_2,\ldots, q_n,\; \pd{S_2}{q_2},\ldots,\pd{S_2}{q_n},\; t \right) = 0 \\
		\det\left( \frac{\partial^2 S_2}{\partial q_i \partial \alpha_j} \right)_{i,j=2,\ldots,n} \neq 0 \\		
		\end{array}\right| \Rightarrow 2.
	\end{flalign*}
	\ovalbox{$\Leftarrow$}
\begin{flalign*}
	& 1., 2. \Rightarrow S = S_1 + S_2 \text{ --- решения уравнения Гамильтона-Якоби.} &\\
	& \frac{dS_1}{dq_1} = \mu(q_1,\; \alpha_1) \qquad f(q_1,\; \mu(q_1,\;\alpha_1)) = \alpha_1 \quad |\; \pd{}{\alpha_1} &\\
	& \pd{f}{\mu}\pd{\mu}{\alpha_1} = 1 \Rightarrow \pd{\mu}{\alpha_1} \neq 0 \Rightarrow \frac{\partial^2 S_1}{\partial q \partial \alpha_1} \neq 0 &\\
	& \det\left( \frac{\partial^2 S_2}{\partial q_i \partial q_j} \right)_{i,j} \neq 0 \Rightarrow \det \frac{\partial^2 S}{\partial q \partial \alpha} \neq 0 &\\ 
\end{flalign*}
\end{proof}

\subsubsection*{Метод разделения переменных}
\begin{enumerate}
	\item \begin{flalign*}
		& H = H(q,\; p) \left( \pd{H}{t} = 0 \right) \Rightarrow S = -h(\alpha)t + S_1(q,\; \alpha): &\\
		& S_1(q,\; \alpha) \text{ --- полный интеграл уравнения } H\left( q,\; \pd{S_1}{q} \right) = h(\alpha) &\\
	\end{flalign*}

	\item \begin{flalign*}
		& H = H(q_2,\ldots,q_n,\; p_1,\ldots,p_n,\; t) \quad \left( \pd{H}{q_1} = 0 \right) \Rightarrow &\\
		& S = \alpha_1q_q + s_2(q_2,\ldots,q_n,\; t)\text{, где } S_2(q_2,\ldots,q_n,\alpha_2,\;t) = 0 \text{ --- полный интеграл } &\\
		& \pd{S_2}{t} + H\left( q_2,\ldots,q_n,\; \alpha_1,\; \pd{S_2}{q_2},\ldots,\pd{S_2}{q_n},\; t \right) = 0 &\\
	\end{flalign*}
	\begin{proof}
		\begin{flalign*}
			& f(q_1,\; p_1) = p_1\; \left( \pd{f}{p_1} \neq 0 \right) \; S = S_1 + S_2 &\\
			& f\left(q_1,\; \pd{S_1}{q_q}\right) = 0\alpha_1 \qquad \frac{dS_1}{dq_1} = \alpha_1 \Rightarrow S_1 = q_1\alpha_1 &\\
		\end{flalign*}
	\end{proof}

	\item \begin{flalign*}
		& H = H(f(q_1\;,p_1),\; q_2,\ldots,q_n,\; p_2,\ldots,p_n) \Rightarrow S = S_1(q_1,\alpha_1) + S_2(q_2,\ldots,q_n,\; \alpha,\; t) &\\
		& \begin{cases}
			f\left( q_1,\; \frac{dS_1}{dq_1} \right) = \alpha_1 \\
			\pd{S_2}{t} + H\left(\alpha,\;q_2,\ldots,q_n,\;\pd{S_2}{q_2},\ldots,\pd{S_2}{q_n},\;t\right) = 0 \\
		\end{cases} &\\
	\end{flalign*}

	\item \begin{flalign*}
		& H = H\left( f_2(f_1(q_1,\; p_1),\; q_2,\; p_2), q_3,\ldots,q_n,\;p_3,\ldots,p_n,\;t \right) &\\
		& S = S_1(q_1,\; \alpha_1) + S_2(q_2,\; \alpha_1,\; \alpha_2) + S_3(q_3,\ldots, q_n,\; \alpha_1,\ldots, \alpha_n,\; t) &\\
		& \begin{cases}
			f_1(q_1,\; \pd{S_1}{q_1}) = \alpha_1 \\
			f_2\left(\alpha_2,\; q_2,\; \pd{S_2}{q_2}\right) = \alpha_2 \\
			\pd{S_3}{t} + H\left( \alpha_2,\; q_3,\ldots,q_n,\;\pd{S_3}{q_3},\ldots,\pd{S_3}{q_n},\; t \right) = 0 \\
		\end{cases} 
	\end{flalign*}
\end{enumerate}

\begin{xmp}
	\begin{flalign*}
		& H = H(f_1(q_1,\; p_1), \ldots,f_m(q_m,\; p_m),\; q_{m + 1}, \ldots q_n,\; p_{m + 1},\ldots, p_n) &\\
		& S = S_1(q_1,\; \alpha_1) + S_2(q_2,\; \alpha) + \ldots + S_m(q_m,\; \alpha_m) + \alpha_{m + 1}q_{m + 1} + \ldots + \alpha_{n - 1}q_{n - 1} - ht + S_0(q_n,\; \alpha_1,\ldots,\alpha_{n - 1},\; h(\alpha)) &\\
		& \begin{cases}
			f_1\left( q_1, \frac{dS_1}{d_1} \right) = \alpha_1 \\
			\ldots \\
			f_m\left( q_m, \frac{dS_m}{q_m} \right) = \alpha_m \\
			-n + H\left( \alpha_1,\ldots,\alpha_m,\; q_n,\; \alpha_{m + 1} \ldots \alpha_{n - 1}\frac{dS_n}{dq_m} \right) &\\
		\end{cases} &\\
		& \frac{dS_1}{dq_1} = F_1(q_1,\; \alpha_1) \Rightarrow S_1 = \int\limits_{q_0}^{q_1}F_1(\xi,\; \alpha_1)d\xi &\\
		& \frac{dS_m}{dq_m} = F_m &\\
		& S_n = \int\limits_{q_m}^{q_n}F_n(\xi,\; \alpha_1,\; \ldots, \alpha_{n - 1},\; h)d\xi &\\
	\end{flalign*}
\end{xmp}